\documentclass[12pt,a4paper]{article}

\usepackage[T2A]{fontenc} %поддержка кириллицы
\usepackage[utf8]{inputenc} %кодировка текста: koi8-r или utf8 в UNIX, cp1251 в Windows
\usepackage[english,russian]{babel} 
\usepackage[left=2cm,right=1.5cm,top=2cm,bottom=2cm]{geometry} 
\usepackage{tabularx} 
\usepackage{graphicx} 
\usepackage{amsmath} %отображение математической нотации
\usepackage{float}
\usepackage{caption, subcaption} %подписи
\usepackage{indentfirst} %отступ вначале параграфа
\parindent 1.27cm %абзацный отступ
%\renewcommand{\baselinestretch}{1.5} %межстрочный интервал
\usepackage{threeparttable}
\captionsetup[table]{labelsep = endash, singlelinecheck=false}
\captionsetup[figure]{name = Рисунок, labelformat=simple, labelsep = endash}

\usepackage{titlesec} 
\titleformat{\section}[block]{\fontsize{16pt}{0cm}\bfseries\filcenter}{\thesection}{1em}{} 
\titleformat{name = \section, numberless}[hang]{\fontsize{16pt}{0cm}\bfseries\filcenter}{\fillast}{1em}{}

\begin{document}

\include{title}
\setcounter{page}{2}

\paragraph{Цель работы:}Изучение математических моделей и исследование характеристик электромеханического объекта управления, построенного на основе электродвигателя постоянного тока независимого возбуждения.
\paragraph{Исходные данные.}
Функциональная схема типичного электромеханического объекта (ЭМО) представлена на рисунке \ref{EMO}. Она включает усилительно-преобразовательное устройство (УПУ), электродвигатель (ЭД), редуктор (Р) и исполнительный механизм (ИМ). \par
Усилительно-преобразовательное устройство служит для формирования напряжения, подаваемого на двигатель в соответствии с управляющим сигналом. Электродвигатель осуществляет преобразование электрической энергии в механическую. Редуктор снижает скорость вращения и повышает момент двигателя на валу ИМ. Для получения информации о состоянии объекта, используемой в устройстве управления, ЭМО снабжено измерительным устройством углового или линейного перемещения (измерители перемещения — ИП).
\begin{figure}[h!]
    \centering
    \includegraphics[width = 0.9\textwidth]{EMO.eps}
    \caption{Функциональная схема ЭМО}
    \label{EMO}
\end{figure}

Исходные данные для выполнения работы приведены в таблице \ref{Tab1}.
\begin{table}[h!]
	\renewcommand{\arraystretch}{1.3} %строки
	\renewcommand{\tabcolsep}{0.3cm} %столбцы
	\centering
	\begin{threeparttable}
    \caption{Исходные данные}
    \begin{tabular}{|c|c|c|c|c|c|c|c|c|c|}
    \hline $U_H,$ & $n_0,$ & $I_H,$ & $M_H,$ & R, & $T_\text{Я},$ & $J_\text{Д},$ & $T_\text{У},$ & $i_p$ & $J_M,$\\
    В & \text{об/мин} & А & Н$\cdot$м & Ом & мc & кг$\cdot$м$^2$ & мс &  & кг$\cdot$м$^2$\\
    \hline 36 &	4000 & 6.5 &	0.57 & 0.85 & 3 & $2.2\cdot10^{-4}$ &	6 &	40 & 0.15\\
    \hline
    \end{tabular} 
    \label{Tab1}
    \end{threeparttable}
\end{table}

\newpage
\section{Расчёт параметров математической модели двигателя}

Произведём расчет необходимых параметров для полной модели:
\begin{gather}
    J_p = 0,2J_\text{Д} = 0.2 \cdot 2.2\cdot10^{-4} = 0.44\cdot10^{-4} [\text{кг}\cdot\text{м}^2]\\
    J_\Sigma = J_\text{Д} + J_p + \frac{J_M}{i_p^2} = 2.2\cdot10^{-4} + 0.44\cdot10^{-4} + \frac{0.15}{40^2} = 3.5775\cdot10^{-4} [\text{кг}\cdot\text{м}^2]\\
    K_E = \frac{U_H}{\omega_0} = \frac{U_H\cdot60}{2\pi\cdot n_0} = \frac{36\cdot60}{2\pi\cdot4000} = 0.8482 [B\cdot c/\text{рад}]\\
    K_m = \frac{M_H}{I_H} = \frac{0.57}{6.5} = 0.08769 [H\cdot\text{м}/A]\\
    K_\text{Д} = \frac{1}{R} = \frac{1}{0.85} = 1.17647 [\text{См}]\\
    K_\text{У} = \frac{U_H}{U_m} = \frac{36}{10} = 3.6 [B]
\end{gather}
\parДля упрощенной модели:
\begin{gather}
	K = \frac{K_\text{У}}{K_E\cdot i_p} = \frac{3.6}{0.8482 \cdot 40} = 0.106107 [\text{рад}/c]\\
	K_f = \frac{R}{K_m\cdot K_E\cdot i_p^2} = \frac{0.85}{0.08769\cdot0.8482\cdot40^2} = 0,0071425 [\text{Ом$\cdot$А$\cdot$рад}/(H\cdot\text{м}\cdot B\cdot c)]\\
	T_M = \frac{R\cdot J_\Sigma}{K_m\cdot K_E} = \frac{0.85\cdot0.00035775}{0.08769\cdot0.8482} = 0.0041[\text{Ом$\cdot$А$\cdot$рад$\cdot$кг$\cdot$м$^2$}/(H\cdot B \cdot c)]
\end{gather}
\parКоэффициенты передачи измерительных устройств $K_U, K_I, K_\omega, K_\alpha$ выбираются таким образом, чтобы обеспечить соответствие максимального значения измеряемого сигнала уровню 10 В на выходе измерительного устройства.\\ 
$K_U = 0.5555555$\\
$K_I = 1.351351351351351$\\
$K_\omega = 0.471222445267512$\\
$K_\alpha = 269.5417789757412$


\newpage
\section{Математическое моделирование полной модели электромеханического объекта}	 
На основе структурной схемы, представленной на рисунке \ref{fullScheme}, составим схему моделирования ЭМО (рисунок \ref{cxema1}).
\begin{figure}[ht!]
	\centering
	\includegraphics[width = \textwidth]{fullScheme.eps}
	\caption{Структурная схема ЭМО}
	\label{fullScheme}
\end{figure}
\begin{figure}[ht!]
	\centering
	\includegraphics[width = \textwidth]{11111.png}
	\caption{Схема моделирования ЭМО}
	\label{cxema1}
\end{figure}

\newpage
Построим графики переходных процессов при $M_{CM}=0$ Н$\cdot$м и U=5B:
\begin{figure}[H]
	\centering
	    \includegraphics[width = 0.9\textwidth]{Napr.eps}
		\caption{Переходный процесс по напряжению}
\end{figure}	
\begin{figure}[H]
	 \centering	
		\includegraphics[width = 0.9\textwidth]{Tok.eps}
		\caption{Переходный процесс по току}
\end{figure}
\begin{figure}[H]	
 \centering
		\includegraphics[width = 0.9\textwidth]{omega.eps}
		\caption{Переходный процесс по угловой скорости}
\end{figure}
\begin{figure}[H]
	 \centering
		\includegraphics[width = 0.9\textwidth]{alpha.eps}
		\caption{Переходный процесс по углу поворота}
\end{figure}
	

\newpage
\section{Исследование влияния момента сопротивления $M_{CM}$ на вид переходных процессов}
Диапазон изменения $M_{CM}$: от 0 Н$\cdot$м до величины, равной $i_pM_H=22.8$. Графики переходных процессов представлены ниже.
\begin{figure}[H]
	\centering
	\includegraphics[width = 0.9\textwidth]{NaprMnog.eps}
	\caption{Переходный процесс по напряжению}
\end{figure}	
\begin{figure}[H]
	 \centering	
	\includegraphics[width = 0.9\textwidth]{TokMnog.eps}
	\caption{Переходный процесс по току}
\end{figure}
\begin{figure}[H]	
	 \centering
	\includegraphics[width = 0.9\textwidth]{omegaMnog.eps}
	\caption{Переходный процесс по угловой скорости}
\end{figure}
\begin{figure}[H]
	 \centering
	\includegraphics[width = 0.9\textwidth]{alphaMnog.eps}
	\caption{Переходный процесс по углу поворота}
\end{figure}

\newpage
\section{Исследование влияния момента нагрузки $J_M$ на вид переходных процессов}
Диапазон изменения момента инерции: $\pm 50\%$  от заданного значения. Графики переходных процессов представлены на рисунках ниже.
\begin{figure}[H]
	\centering
	\includegraphics[width = 0.9\textwidth]{Naprjjj.eps}
	\caption{Переходный процесс по напряжению}
\end{figure}	
\begin{figure}[H]
	 \centering	
	\includegraphics[width = 0.9\textwidth]{Tokjjj.eps}
	\caption{Переходный процесс по току}
\end{figure}
\begin{figure}[H]	
	 \centering
	\includegraphics[width = 0.9\textwidth]{omegajjj.eps}
	\caption{Переходный процесс по угловой скорости}
\end{figure}
\begin{figure}[H]
	 \centering
	\includegraphics[width = 0.9\textwidth]{alphajjj.eps}
	\caption{Переходный процесс по углу поворота}
\end{figure}

\newpage
\section{Исследование передаточного отношения редуктора $i_p$ на вид переходных процессов}
Исследования проводились при величине момента сопротивления $M_{CM}=0$ и при $M_{CM}=11.4$. Диапазон изменения передаточного отношения: $\pm 75\%$  от заданного значения.

Графики переходных процессов, при $M_{CM}=0$: 
\begin{figure}[H]
	\centering
	\includegraphics[width = 0.9\textwidth]{Napriii0.eps}
	\caption{Переходный процесс по напряжению}
\end{figure}	
\begin{figure}[H]	
	 \centering
	\includegraphics[width = 0.9\textwidth]{Tokiii0.eps}
	\caption{Переходный процесс по току}
\end{figure}
\begin{figure}[H]	
	 \centering
	\includegraphics[width = 0.7\textwidth]{omegaiii0.eps}
	\caption{Переходный процесс по угловой скорости}
\end{figure}
\begin{figure}[H]
	 \centering
	\includegraphics[width = 0.7\textwidth]{alphaiii0.eps}
	\caption{Переходный процесс по углу поворота}
\end{figure}
Графики переходных процессов, при $M_{CM}=11.4$:
\begin{figure}[H]
	\centering
	\includegraphics[width = 0.7\textwidth]{Napriii114.eps}
	\caption{Переходный процесс по напряжению}
\end{figure}	
\begin{figure}[H]
	 \centering	
	\includegraphics[width = 0.7\textwidth]{Tokiii114.eps}
	\caption{Переходный процесс по току}
\end{figure}
\begin{figure}[H]	
	 \centering
	\includegraphics[width = 0.7\textwidth]{omegaiii114.eps}
	\caption{Переходный процесс по угловой скорости}
\end{figure}
\begin{figure}[H]
	 \centering
	\includegraphics[width = 0.7\textwidth]{alphaiii114.eps}
	\caption{Переходный процесс по углу поворота}
\end{figure}

\newpage
\section{Исследование влияния постоянных времени на вид переходных процессов}
Исследования проводились при значениях постоянных времени $T_\text{у} = \frac{0.6}{10} \text{мс} = 0.0006$c, $T_\text{я} = \frac{0.3}{10} \text{мс} = 0.0003$c. Графики переходных процессов изображены на рисунках ниже.
\begin{figure}[H]
	\centering
	\includegraphics[width = 0.9\textwidth]{NaprTT.eps}
	\caption{Переходный процесс по напряжению}
\end{figure}	
\begin{figure}[H]	
	 \centering
	\includegraphics[width = 0.9\textwidth]{TokTT.eps}
	\caption{Переходный процесс по току}
\end{figure}
\begin{figure}[H]	
	 \centering
	\includegraphics[width = 0.9\textwidth]{omegaTT.eps}
	\caption{Переходный процесс по угловой скорости}
\end{figure}
\begin{figure}[H]
	 \centering
	\includegraphics[width = 0.9\textwidth]{alphaTT.eps}
	\caption{Переходный процесс по углу поворота}
\end{figure}

Из этих графиков видно, что при уменьшении значений постоянных времени на порядок установившиеся значения тока и скорости, а также время переходного процесса этих величин не изменились, однако возросло максимальное значение тока.

\newpage
\section{Математическое моделирование упрощённой модели электромеханического объекта}	 
На основе структурной схемы, представленной на рисунке \ref{simpScheme}, составим схему моделирования ЭМО (рисунок \ref{cxema2}).
\begin{figure}[ht!]
	\centering
	\includegraphics[width = \textwidth]{simpScheme.eps}
	\caption{Структурная схема ЭМО}
	\label{simpScheme}
\end{figure}
\begin{figure}[ht!]
	\centering
	\includegraphics[width = \textwidth]{cxema222.png}
	\caption{Схема моделирования ЭМО}
	\label{cxema2}
\end{figure}

Для того, чтобы проанализировать погрешности проведём сравнение графиков переходных процессов по угловой скорости полной и упрощённой модели ЭМО, а также полной модели при меньших значениях постоянных времени и упрощённой модели (рисунок \ref{sravnenie}). 
\begin{figure}[H]	
	 \centering
	\includegraphics[width = 0.9\textwidth]{omega2mod.eps}
	\caption{Переходный процесс по угловой скорости}
\end{figure}
\begin{figure}[H]
	 \centering
	\includegraphics[width = 0.9\textwidth]{alpha2mod.eps}
	\caption{Переходный процесс по углу поворота}

	\label{sravnenie}
\end{figure}

\newpage
\section{Вывод математических моделей вход-состояние-выход для полной и упрощенной схем моделирования ЭМО}
Полная модель ЭМО.\par
Для составления математической модели запишем формулы, характеризующие ЭМО, взятые из теории к данной лабораторной работе.
\begin{align}
	&&\begin{cases}
		T_\text{Я}\dot{I} + I = K_\text{Д}(U_\text{У} - K_E\omega)\\
		M_\text{Д} - M_C = J_\Sigma\dot{\omega}\\
		\dot{\alpha} = \omega\\
		T_\text{У}\dot{U_\text{У}} + U_\text{У} = K_\text{У}U
	\end{cases}
	\Rightarrow
	\begin{cases}
		\dot{I} = \displaystyle{- \frac{1}{T_\text{Я}}I + \frac{K_\text{Д}}{T_\text{Я}}U_\text{У} -\frac{K_E}{T_\text{Я}}}\omega \\
		\dot{\omega} = \displaystyle{\frac{K_m}{J_\Sigma}}I - \frac{1}{J_\Sigma}M_C\\
		\dot{\alpha} = \omega\\
		\dot{U_\text{У}} = -\displaystyle{\frac{1}{T_\text{У}}U_\text{У}} + \frac{K_\text{У}}{T_\text{У}}U
	\end{cases}
	,
	\label{ESETh}
\end{align}
где $M_\text{Д} = K_mI$.
	
Примем вектор состояния
$
	X =
	\begin{bmatrix}
		\alpha & \omega & I & U_\text{У}
	\end{bmatrix}^T
$
и вектор входных воздействий
$
	U =
	\begin{bmatrix}
		U & M_C
	\end{bmatrix}^T,
$
тогда исходя из (\ref{ESETh}) получим модель Вход-Состояние-Выход:
\begin{align}
	\begin{cases}
		\dot{X} = AX + BU\\
		y = CX
	\end{cases} \Rightarrow
	\begin{cases}
		\begin{bmatrix}
			\dot{\alpha}\\
			\dot{\omega}\\
			\dot{I}\\
			\dot{U_\text{У}}
		\end{bmatrix} =
		\begin{bmatrix}
			0 & 1 & 0 & 0\\
			0 & 0 & \displaystyle{\frac{K_m}{J_\Sigma}} & 0\\
			0 & -\displaystyle{\frac{K_E}{T_\text{Я}}} & -\displaystyle{\frac{1}{T_\text{Я}}} & \displaystyle{\frac{K_\text{Д}}{T_\text{Я}}}\\
			0 & 0 & 0 & -\displaystyle{\frac{1}{T_\text{У}}}
		\end{bmatrix}
		\begin{bmatrix}
			\alpha\\
			\omega\\
			I\\
			U_\text{У}
		\end{bmatrix}
		+
		\begin{bmatrix}
			0 & 0\\
			0 & -\displaystyle{\frac{1}{J_\Sigma}}\\
			0 & 0\\
			\displaystyle{\frac{K_\text{У}}{T_\text{У}}} & 0
		\end{bmatrix}
		\begin{bmatrix}
			U\\
			M_C
		\end{bmatrix}\\
		\alpha = 
		\begin{bmatrix}
			1 & 0 & 0 & 0
		\end{bmatrix}
		\begin{bmatrix}
			\alpha\\
			\omega\\
			I\\
			U_\text{У}
		\end{bmatrix}.
	\end{cases}
	\label{ESEFull}
\end{align}
	
Подставив рассчитанные ранее значения, получим следующие матрицы
\begin{align}
	&&A = 
	\begin{bmatrix}
		0 & 1 & 0 & 0\\
		0 & 0 & 36.33 & 0\\
		0 & -92 & -200 & 266\\
		0 & 0 & 0 & -166,67
	\end{bmatrix}, B =
	\begin{bmatrix}
		0 & 0\\
		0 & -78,99\\
		0 & 0\\
		800 & 0
	\end{bmatrix}
\end{align}

Упрощенная модель.\par
Для составления упрощённой модели ЭМО постоянные времени $T_\text{У}$ и $T_\text{Я}$ приравнивают к 0, так как их значение существенно меньше, чем значение механической постоянной времени $T_M$. Для получения упрощённой модели Вход-Состояние-Выход произведём соответствующие подстановки в уравнения для полной системы (\ref{ESETh}).
\begin{align}
	&&\begin{cases}
		\dot{\omega} = -\displaystyle{\frac{K_MK_\text{Д}K_E}{J_\Sigma}}\omega + \frac{K_MK_\text{Д}K_E}{J_\Sigma}U - \frac{1}{J_\Sigma}M_C\\
		\dot{\alpha} = \omega
	\end{cases},
\end{align}
и на основании полученной системы построим модель:
\begin{align}
	&&\begin{cases}
		\begin{bmatrix}
			\dot{\alpha}\\
			\dot{\omega}
		\end{bmatrix} =
		\begin{bmatrix}
			0 & 1\\
			0 & -\displaystyle{\frac{K_MK_\text{Д}K_E}{J_\Sigma}}
		\end{bmatrix}
		\begin{bmatrix}
			\alpha\\
			\omega
		\end{bmatrix} + 
		\begin{bmatrix}
			0 & 0\\
			\displaystyle{\frac{K_MK_\text{Д}K_E}{J_\Sigma}} & -\displaystyle{\frac{1}{J_\Sigma}}
		\end{bmatrix}
		\begin{bmatrix}
			U\\
			M_C
		\end{bmatrix}\\
		\alpha =
		\begin{bmatrix}
			1 & 0
		\end{bmatrix}
		\begin{bmatrix}
			\alpha\\
			\omega
		\end{bmatrix}
	\end{cases}.
\end{align}
	
Подставив значения, получим матрицы:
\begin{align}
	&&A =
	\begin{bmatrix}
		0 & 1\\
		0 & -22,23
	\end{bmatrix}, B = 
	\begin{bmatrix}
		0 & 0\\
		22,23 & -78,99
	\end{bmatrix}
\end{align}

\newpage
\section*{Вывод}

В ходе лабораторной работы было проведено исследование математических моделей электромеханического объекта управления. Были выявлены изменения в переходных процессах системы путём изменения таких параметров как момент сопротивления, момент нагрузки, передаточное отношение редуктора.\par
Как видно из рисунков 8-11 при увеличении момента сопротивления установившееся значение тока якоря увеличивается, а значение угловой скорости уменьшается. Время переходного процесса по току уменьшается с 0.24с до 0.21с, а по скорости остается постоянным и равным 0.2с.\par
При исследовании момента инерции нагрузки было показано, что его увеличение ведёт к возрастанию времени переходного процесса по угловой скорости и по току, в то время как установившееся значение этих двух параметров остается неизменным, что можно увидеть на графиках, изображенных на рисунках 12-15.\par
Так же можно наблюдать, что в случае нулевого момента сопротивления при увеличении передаточного отношения редуктора максимальное значение тока и время переходного процесса уменьшаются. Установившиеся значения тока и угловой скорости при этом остаются постоянными.\par
В случае момента сопротивления равном половине максимального значения при увеличении $i_p$ не только уменьшается время переходного процесса по току и скорости, а также установившееся значение тока, но и увеличивается установившееся значение скорости.\par
Как можно заметить по результатам математического моделирования при постоянных времени много меньших по сравнению с механической постоянной времени переходной процесс по скорости и углу поворота ротора не имеет значительных изменений, поэтому ими можно пренебречь и перейти к упрощенной модели.

\end{document}
