\documentclass[12pt,a4paper]{article}

\usepackage[T2A]{fontenc} %поддержка кириллицы
\usepackage[utf8]{inputenc} %кодировка текста: koi8-r или utf8 в UNIX, cp1251 в Windows
\usepackage[english,russian]{babel} 
\usepackage[left=2cm,right=1.5cm,top=2cm,bottom=2cm]{geometry} 
\usepackage{tabularx} 
\usepackage{graphicx} 
\usepackage{amsmath} %отображение математической нотации
\usepackage{float}
\usepackage{caption, subcaption} %подписи
\usepackage{indentfirst} %отступ вначале параграфа
\parindent 1.27cm %абзацный отступ
%\renewcommand{\baselinestretch}{1.5} %межстрочный интервал
\usepackage{threeparttable}
\captionsetup[table]{labelsep = endash, singlelinecheck=false}
\captionsetup[figure]{name = Рисунок, labelformat=simple, labelsep = endash}

\usepackage{titlesec} 
\titleformat{\section}[block]{\fontsize{16pt}{0cm}\bfseries\filcenter}{\thesection}{1em}{} 
\titleformat{name = \section, numberless}[hang]{\fontsize{16pt}{0cm}\bfseries\filcenter}{\fillast}{1em}{}

\begin{document}

\include{title}
\setcounter{page}{2}

\paragraph{Цель работы:}Изучение математических моделей и исследование характеристик исполнительного устройства, построенного на основе пьезоэлектрического двигателя микроперемещений.
\paragraph{Исходные данные.}

Исходные данные для выполнения работы приведены в таблице \ref{Tab1}.
\begin{table}[h!]
	\renewcommand{\arraystretch}{1.3} %строки
	\renewcommand{\tabcolsep}{0.3cm} %столбцы
	\centering
	\begin{threeparttable}
    \caption{Исходные данные}
    \begin{tabular}{|c|c|c|c|c|c|c|c|}
    \hline $C_P$ & m & $K_O$ & $K_d$ & $T_u$ & $F_B$ & $U_{Pm}$ & $U_m$\\
    Н/м & кг & H/B & Нc/м & мc & H & B & B\\
    \hline $0.8\cdot10^8$ & 0.5 & 9.3 & $0.8\cdot10^3$ & 0.08 & 75 & 300 & 10\\
    \hline
    \end{tabular} 
    \label{Tab1}
    \end{threeparttable}
\end{table}

$K_u=U_{Pm}/U_m=300/10=30$
\parКоэффициенты передачи $K_u^{-1}, K_F, K_V, K_X$ определяются так, чтобы обеспечить соответствие максимального значения измеряемого сигнала уровню 10 В на выходе измерительного устройства.\\ 
$K_u^{-1} = 0.03333$\\
$K_F = 0.00769$\\
$K_V = 33.33333$\\
$K_X = 183486.238532$

\newpage
\section{Математическое моделирование модели пьезоэлектрического исполнительного устройства}	 
На основе структурной схемы, представленной на рисунке \ref{structScheme}, составим схему моделирования ПД (рисунок \ref{cxema1.png}).
\begin{figure}[ht!]
	\centering
	\includegraphics[width = 0.8 \textwidth]{structScheme}
	\caption{Структурная схема пьезоэлектрического исполнительного устройства}
	\label{structScheme}
\end{figure}
\begin{figure}[ht!]
	\centering
	\includegraphics[width = \textwidth]{cxema1.png}
	\caption{Схема моделирования ПД}
	\label{cxema1.png}
\end{figure}

\newpage
Построим графики переходных процессов при $F_B=0$H и U=10B:
\begin{figure}[H]
	\centering
		\centering
		\includegraphics[width = 0.7\textwidth]{1U.eps}
		\caption{График переходного процесса напряжения}
\end{figure}
\begin{figure}[H]		
		\centering
		\includegraphics[width = 0.7\textwidth]{1F.eps}
		\caption{График переходного процесса cилы}
\end{figure}
\begin{figure}[H]	
		\centering
		\includegraphics[width = 0.7\textwidth]{1V.eps}
	    \caption{График переходного процесса скорости}
\end{figure}	
\begin{figure}[H]
		\centering
		\includegraphics[width = 0.9\textwidth]{1X.eps}
		\caption{График переходного процесса координаты}
\end{figure}	
	

\newpage
\section{Исследование влияния массы нагрузки m на вид переходных процессов}
Диапазон изменения массы нагрузки m: $\pm 50\%$  от заданного значения. Графики переходных процессов представлены на рисунке \begin{figure}[H]
	\centering
	\centering
	\includegraphics[width = 0.7\textwidth]{2U.eps}
	\caption{Графики переходных процессов напряжения}
\end{figure}
\begin{figure}[H]		
	\centering
	\includegraphics[width = 0.7\textwidth]{2F.eps}
	\caption{Графики переходных процессов cилы}
\end{figure}
\begin{figure}[H]	
	\centering
	\includegraphics[width = 0.7\textwidth]{2V.eps}
	\caption{Графики переходных процессов скорости}
\end{figure}	
\begin{figure}[H]
	\centering
	\includegraphics[width = 0.9\textwidth]{2X.eps}
	\caption{Графики переходных процессов координаты}
\end{figure}

По временным диаграммам определим время переходного процесса $t_\text{П}$, величину перерегулирования $\sigma$ и установившееся значение $X_y$. Занесём результаты в таблицу \ref{Tab2}.
\begin{table}[h!]
	\renewcommand{\arraystretch}{1.3} %строки
	\renewcommand{\tabcolsep}{0.3cm} %столбцы
	\centering
	\begin{threeparttable}
    \caption{Характеристики системы при меняющейся массе нагрузки}
    \begin{tabular}{|c|c|c|c|}
    \hline m, \text{кг} & $t_\text{П}, \text{мс}$ & $\sigma, \%$ & $X_y$\\
    \hline 0.75 & 4.7 & 64.4 & 6.4 \\
    \hline 0.50 & 3.3 & 56.3 & 6.4 \\
    \hline 0.25 & 1.6 & 37.5 & 6.4 \\
    \hline
    \end{tabular} 
    \label{Tab2}
    \end{threeparttable}
\end{table}

\newpage
\section{Исследование влияния $T_u$ на вид переходных процессов}
Изменение $T_u$ в сторону увеличивая исходного значения постоянной времени в 2, 4 и 6 раз. Графики переходных процессов:
\begin{figure}[H]
	\centering
	\centering
	\includegraphics[width = 0.7\textwidth]{3U.eps}
	\caption{Графики переходных процессов напряжения}
\end{figure}
\begin{figure}[H]		
	\centering
	\includegraphics[width = 0.7\textwidth]{3F.eps}
	\caption{Графики переходных процессов cилы}
\end{figure}
\begin{figure}[H]	
	\centering
	\includegraphics[width = 0.7\textwidth]{3V.eps}
	\caption{Графики переходных процессов скорости}
\end{figure}	
\begin{figure}[H]
	\centering
	\includegraphics[width = 0.9\textwidth]{3X.eps}
	\caption{Графики переходных процессов координаты}
\end{figure}

По результатам моделирования определим время переходных процессов $t_\text{П}$, величину перерегулирования $\sigma$ и установившееся значение $X_y$. Занесём результаты в таблицу \ref{Tab3}.
\begin{table}[h!]
	\renewcommand{\arraystretch}{1.3} %строки
	\renewcommand{\tabcolsep}{0.3cm} %столбцы
	\centering
	\begin{threeparttable}
    \caption{Характеристики системы при меняющейся постоянной времени}
    \begin{tabular}{|c|c|c|c|c|c|c|}
    \hline $T_u$, \text{мс} & $t_\text{П}, \text{мс}$ & $\sigma, \%$ & $X_y$ & $s_1$ & $s_2$ & $s_3$\\
    \hline 0.16 & 2.4 & 24.875  &  6.4 & -6250 & -800-12623.8i & -800+12623.8i \\
    \hline 0.32 & 1.9 & 6.82   &  6.4 & -3125 & -800-12623.8i & -800+12623.8i\\ 
    \hline 0.48 & 1.6 & 1.748   &  6.4 & -2083.33 & -800-12623.8i & -800+12623.8i\\ 
    \hline
    \end{tabular} 
    \label{Tab3}
    \end{threeparttable}
\end{table}

Чтобы рассчитать значения корней характеристического уравнения получим передаточную функцию. Для этого будем рассматривать исполнительное пьезоэлектрическое устройство как упругую механическую систему. В этом случае математическая модель может быть получена на основе уравнения баланса сил в пьезодвигателе:  
\begin{equation} 
    F_y = F_O + F_\text{Д} + F_d + F_B,
    \label{F}
\end{equation}
где $F_y=C_px$ --- усилие упругой деформации ПД, $F_O=K_OU_p$ --- усилие, вызванное обратным пьезоэффектом, $F_\text{Д}=-m\displaystyle{\frac{d^2x}{dt^2}}$ --- динамическое усилие в ПД, $F_d=-K_d\displaystyle{\frac{dx}{dt}}$ --- демпфирующее усилие, обусловленное механическими потерями, $F_B$ --- внешнее воздействие, x --- перемещение, $C_p$ --- коэффициент упругости, $K_O$ --- коэффициент обратного пьезоэффекта, $U_p$ --- напряжение на электродах ПД, m --- масса перемещаемой нагрузки, $K_d$ --- коэффициент демпфирования.\par
Подставив перечисленные равенства в уравнение (\ref{F}), получим:
\begin{equation} 
    m\ddot{x} + K_d\dot{x} + C_px = K_OU_p + F_B
    \label{F1}
\end{equation}
\par Составленная по уравнению (\ref{F1}) передаточная функция будет выглядеть следующем образом:
\begin{equation} 
    W_{\text{ВУ}}(s)=\frac{K_OU_p + F_B}{ms^2 + K_ds + C_p}
    \label{FVU}
\end{equation}
\par Управление ПД осуществляется от высоковольтного усилителя, который, в нашем случае, описывается апериодическим звеном первого порядка:
\begin{equation} 
    W(s)=\frac{K_u}{T_us + 1}
\end{equation}
\par Исходя из того, что ВУ и ПД соединены последовательно, имеем передаточную следующую функцию:
\begin{equation} 
    W(s)=\frac{K_u(K_OU_p + F_B)}{(T_us + 1)(ms^2 + K_ds + C_p)}
\end{equation}
\par Найдем корни характеристического уравнения для всех сочетаний параметров и запишем результат в таблицу \ref{Tab3}.

\newpage
\section{Исследование влияния коэффициента упругости $C_p$ на вид переходных процессов}
Исследования проводились при значениях коэффициента упругости 0.5$C_p$ и 2$C_p$ при $F_B=80$H и U=0B. Графики переходных процессов:
\begin{figure}[H]	
	\centering
	\includegraphics[width = 0.9\textwidth]{4V.eps}
	\caption{Графики переходных процессов скорости}
\end{figure}	
\begin{figure}[H]
	\centering
	\includegraphics[width = 0.9\textwidth]{4X.eps}
	\caption{Графики переходных процессов координаты}
\end{figure}



\newpage
\section{Построение асимптотической ЛАЧХ исполнительного устройства}
Представим передаточную функцию (\ref{FVU}) в виде колебательного звена:
\begin{equation} 
    W(s) = \frac{\displaystyle{\frac{K_0}{C_p}}}{\displaystyle{\frac{m}{C_p}}s^2 + \frac{K_d}{C_p}s + 1}.
\end{equation}

Асимптотическая логарифмическая амплитудная характеристика будет иметь нулевой наклон на уровне 
\begin{equation} 
   20\lg\displaystyle{\frac{K_0}{C_p}} = 20\lg \displaystyle{\frac{9.3}{0.8\cdot10^8}} = -138.692 \text{дБ}
\end{equation}
до сопрягающей частоты 
\begin{equation} 
	\omega_c = \sqrt{\displaystyle{\frac{C_p}{m}}} = \sqrt{\displaystyle{\frac{0.8\cdot10^8}{0.5}}} = 12649.11 \text{рад/с}.
\end{equation}
\parПосле сопрягающей частоты график пойдёт под наклоном в -40 дБ/дек. Таким образом асимптотическая ЛАЧХ будет выглядить так как показано на рисунке \ref{L}:
\begin{figure}[H]
	\centering
	\includegraphics[width = \textwidth]{L.png}
	\caption{Асимптотическая ЛАЧХ исполнительного устройства}
	\label{L}
\end{figure}

\newpage
\section*{Вывод}
В ходе лабораторной работы было проведено исследование пьезоэлектрического устройства. 
Были выявлены изменения в переходных процессах системы путём изменения таких параметров как масса нагрузки, постоянная времени, коэффициент упругости.\par
Как видно из таблицы 2 при уменьшении массы нагрузки установившееся значение перемещения остаётся постоянным, а значение времени переходного процесса и перерегулирования уменьшается. \par
При исследовании влияния постоянной времени вольтного усилителя было показано, что её увеличение ведёт к уменьшению перерегулирования, а также к уменьшению одного из корней характеристического уравнения, что можно увидеть в таблице 3.\par
Из графиков видно, что при увеличении значения коэффициента упругости пьезоэлемента увеличивается установившееся значение перемещения пьезокерамических пластин.

\end{document}
